1. How to control hands in VR? Focused area, in-depth exploration
2. How are hands controlled normally? -> Job Sim
3. Problem identification -> hands go into objects
4. Exploration of the problem through prototyping
"Prototypes are design-thinking enablers not just tools for evaluating or proving successes of failures of design outcomes"
5. Conflict identification: Virtual Constraints Problem
6. Controlled design experiments: within-subject comparison of two setups (baseline jobsim and example adjusted hand control model). Discuss problems mentioned by \parencite{Waern2015}
7. 


\todo[inline]{We want some methodology for the experimental and test parts.
Ideas:
"When researchers actually construct something, they find problems and discover things that would otherwise go unnoticed. These observations unleash wisdom, countering a typical academic tendency to value thinking and discourse over doing." (Koskinen et al., 2011, p. 2)
When working in this constructive way, researchers make ideas tangible in order to be able to discuss and critique them.
The Anatomy of Prototypes: Prototypes as Filters, Prototypes as Manifestations of Design Ideas (Lim, Stolterman and Tenenberg, 2008) -> Prototypes not only for evaluation: 
"They [the prototypes] are design-thinking enablers deeply embedded and immersed in design practice and not just tools for evaluating or proving successes or failures of design outcomes" (Lim, Stolterman, and Tenenberg, 2008, p. 2) 
and
"In design and development processes, prototypes are used not for providing solutions but for discovering problems or for exploring new solution directions." (Lim, Stolterman, and Tenenberg, 2008, p. 8)
Experimental Design \parencite{Waern2015}
Qualitative Research
Hermeneutic Spiral?}