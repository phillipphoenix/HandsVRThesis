Our method of research is based on prototyping as a way to gain knowledge about our research topic, in Koskinen et al.'s words:

\begin{displayquote}
\textit{"When researchers actually construct something, they find problems and discover things that would otherwise go unnoticed. These observations unleash wisdom, countering a typical academic tendency to value thinking and discourse over doing."} \parencite{Koskinen2011}
\end{displayquote}

We use prototypes as "design-thinking enablers", "to discover problems and explore new solution directions" \parencite{Lim2008}, but also as tools to evaluate and show the potential of different techniques to deal with a specific problem: the Virtual Constraints Problem.

Although we do not prototype games, our main area of interest is games and the ways of controlling hands in VR that we prototype would constitute game mechanics, understood as "methods invoked by agents for interacting with the game world" \parencite{Sicart2008}, in a wide range of VR games.

Our process resembles the experimental game design research approach described by \parencite{Waern2015}. Through the thought-enabling process of developing prototypes, we developed our understanding of an initially less formally defined problem which was originally identified through observation of current VR games.

Using the prototypes we created, we conducted a controlled experiment \parencite{Waern2015} to compare an approach to hand control found in existing commercial games with an alternative version that partly deals with the problem under scrutiny in order to assess whether the direction we explored in our prototypes shows promise and should be further explored.

Waern et al. warn about common pitfalls of controlled experiments in experimental game design research, but the nature of our area of focus make the controlled experiments approach appropriate. The first of the problems they point out is that when trying to compare variants of a game, having to develop a game for this purpose entails a great overhead, but our work is limited to a very specific mechanic that can be tested with minimal context, so developing a full game was in principle not necessary.

For the same reason, although comparing variants of a core mechanic in a game could indeed be problematic since games are built around their core mechanics and thus, either the other features of the game remain constant, and then the comparison might be unfair because the other features might work better with one of the variants, or if the other features of the game aren't constant, they can easily become confounding factors in the evaluation.

Finally, results obtained with tests in one single game are not necessarily generalizable. In this regard, we clarify that our intent is not to prove one variant to be superior to another, but the asses whether one of the variants can be useful for VR games and other VR applications where "presence" is regarded as a desirable quality.

Using the gained insight on the problem, we found connections to an existing corpus of research that justifies further research in the direction that was taken and provides us with vocabulary, structure and concepts that can be used to analyze the problem.\footnote{Despite this connection to previous academic research having been found late in the process, we present it first for the convenience of the reader.}