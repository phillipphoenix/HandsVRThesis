In the previous chapter the terms "immersion", "presence" (and "hand presence") and even "virtual reality" are used in a loose manner. In academia, however, these terms are problematic because of the wide range of fields in which they are used, with very specific and not always similar meanings. Here we will frame the virtual environment constraints problem in existing theories and ground these terms in the academic literature.

\section{Definitions}
\label{sec:definitions}

\subsection{Virtual Reality}
\label{subsec:vrDef}

Virtual Reality is frequently thought of in terms of a collection of hardware devices commonly associated with the medium: head mounted displays, tracking devices, etc... One such definition focus on the technology is Coates':

\begin{displayquote}
\textit{"Virtual Reality is electronic simulations of environments experienced via head mounted eye goggles and wired clothing enabling the end user to interact in realistic three-dimensional situations."} As cited in \parencite{Steuer1992}.
\end{displayquote}

However, for the purposes of various fields of knowledge such as communication research and for software developers, \parencite{Steuer1992} argues that the medium needs a definition with an experiential focus instead of a technological one. Steuer works his way from Gibson's definition of presence "the sense of being in an environment" (as cited in \parencite{Steuer1992}) through telepresence as "the experience of presence in an environment by means of a communication medium" \parencite{Steuer1992} and finally arrives at a definition of virtual reality independent of any technological specifics as:

\begin{displayquote}
\textit{A virtual reality is a real or simulated environment in which a perceiver experiences a sense of being in that environment by means of a communication medium.} \parencite{Steuer1992}
\end{displayquote}

\subsection{Immersion}
\label{subsec:immersion}

Slater provides a definition of immersion based on "the actions we know to carry out in order to perceive" or Sensorimotor Contingencies (SC), from behavioural and brain scientist O'Regan and Noë (as cited in \parencite{Slater2009}). Examples of SCs are turning one's moving your eyes to look in a certain direction, reaching out with our hand to feel a surface with our sense of touch or turning our head sideways to align one of our ears with the direction that a sound seems to come from in order to hear it more clearly. Note that SCs relate to all of our senses. Slater's definition of immersion is:

\begin{displayquote}
\textit{Immersion is a property of the actions that the user can take that result in changes in perception (Valid Sensorimotor Actions), or changes to the environment (Valid Effectual Actions) within a system.} \parencite{Slater2009}
\end{displayquote}

This means that immersion is an objective property of virtual reality systems that can be used to compare them. A system can be said to be more immersive than another system if the first system's set of Valid Actions (Valid Sensorimotor Actions + Valid Effectual Actions) is greater than the second system's set.

\subsection{Place Illusion and Plausibility Illusion \textit{in lieu of} Presence}
\label{subsec:PIandPsi}

In the context of Virtual Reality, the term presence is usually used meaning the "sensation of being in the virtual world" \parencite{Schuemie2001}. To avoid possible confusion due to the multiple definitions and theories related to the term "presence", we will instead of it use the terms Place Illusion (PI) and Plausibility Illusion (Psi), proposed in \parencite{Slater2009}, where they are defined as follows:

\begin{displayquote}
\textit{"Place Illusion is the illusion of being in a place in spite of the sure knowledge that you are not there." }
\end{displayquote}

\begin{displayquote}
\textit{"Plausibility Illusion is the illusion that what is apparently happening is really happening in spite of the knowledge that it is not."}
\end{displayquote}

PI results from "using your body to perceive in the way that you would normally" \parencite{Slater2015}, it is bounded by the Valid Sensorimotor Actions in a system and therefore by how immersive a system is.

On the other hand, Psi has a more cognitive nature and is the result of the perceived coherence of the virtual world and the extent to which its reactions to your actions meet your expectations about how reality works \parencite{Slater2015}.

Slater frequently uses the example of virtual characters acknowledging you by looking at you and smiling or by talking to you as an event that can produce Psi. Virtual events that you don't control and that refer directly to you include you in the virtual world and because you are real, when your inclusion in the virtual world makes it seem more real.

Interestingly, Slater doesn't seem to associate being able to use your virtual body to affect the environment to PI. This could however be seen to fall under the scope of Psi, since one would expect to be able to manipulate virtual objects that resemble real objects in the same way we manipulate them in reality. \todo{Move this to next section?}

\todo{Why do we want Psi and PI? Slater says when they happen, the user reacts to situations like in real life.}

\subsection{Sense of Embodiment}
\label{subsec:embodiment}

In dealing with virtual hands, we are considering what connection one might experience towards a virtual body. \parencite{Kilteni2012} propose the notion of Sense of Embodiment (SoE) to address the issue of "whether it is possible to experience the same sensations towards a virtual body as toward a biological body":

\begin{displayquote}
\textit{SoE toward a body B is the sense that emerges when B's properties are processed as if they were the properties of one's own biological body.}
\end{displayquote}

Kilteni et al. further explain SoE in terms of three subcomponents:

\begin{displayquote}
\begin{itemize}
\item \textit{Sense of Self-Location: one's spacial experience of being inside a body.}
\item \textit{Sense of Agency: "the sense of having global motor control, including the subjective experience of action, control, intention, motor selection and the consious experience of will"} (Blanke and Metzinger's definition, as cited in \parencite{Kilteni2012})\textit{, resulting from the comparison between the predicted sensory consequences of one's actions and the actual sensory consequences.}
\item \textit{Sense of Body Ownership: one's self attribution of the body, implying the sense that the body is the source of the experienced sensations.}
\end{itemize}
\end{displayquote}

\section{The Virtual Constraints Problem}
\label{sec:virtualContraintsProblem}

We now seek to describe the problem that we are addressing, that is, the Virtual Constraints Problem, using the concepts that we have just defined. Furthermore, we shall use the theory to argue why the adjusted hand control model is an approach worth exploring.

The virtual constraints problem has two conflicting goals:

\begin{enumerate}
\item Make the virtual body respect the virtual environment physics.
\item Preserve the connection between the user's real body and the virtual body.
\end{enumerate}

The first goal is related to Psi. The user knows how real bodies interact with the environment and if their virtual body does not interact with the environment like they know real bodies do, this dissonance will remind the user of the irreality of the virtual experience. If the world we are perceiving doesn't work like the real world, the world we are perceiving cannot be the real world, thus what is apparently happening is not really happening, which implies a break of Psi.

The second goal has to do with PI and SoE.

\section{TODO}

\begin{itemize}
\item Map terms to proposed desirable properties of virtual hands.
\item Fidelity coherence and expectations \parencite{Nowak2003, Argelaguet2016}
\item Address Norman's critique of movement based interaction \parencite{Gillies2016}
\item Rubber hand illusion and adjusted hand control model \parencite{Sanchez-Vives2010}, \parencite{Fourneret1998}
\item presence / place illusion / SoE with different sensorimotor contingencies or not in the same capacity as we are present in the real world
\item Oculus Touch vs Data Gloves -> presence as involvement (allocation of attentional resources)
\end{itemize}