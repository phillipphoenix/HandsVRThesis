In the previous chapter the terms "immersion", "presence" (and "hand presence") and even "virtual reality" are used in a loose manner. In academia, however, these terms are problematic because of the wide range of fields in which they are used, with very specific and not always similar meanings. Here we will frame the virtual environment constraints problem in existing theories and ground these terms in the academic literature.

\section{Definitions}
\label{sec:definitions}

\subsection{Virtual Reality}
\label{subsec:vrDef}

Virtual Reality is frequently thought of in terms of a collection of hardware commonly associated with the term: head mounted displays, tracking devices, etc... \parencite{Steuer1992} argues that the medium needs a definition with an experiential focus, instead of a technological one, for the purposes of various fields of knowledge such as communication researchers and software developers.

Steuer proposes a definition of Virtual Reality based on the concepts of presence and telepresence. Presence is described by Gibson as "the sense of being in an environment" (cited as found in \parencite{Steuer1992}) and in turn Steuer defines telepresence as "the experience of presence in an environment by means of a communication medium". From these terms we can rewrite Steuer's definition of Virtual Reality as follows:

\begin{displayquote}
\textit{A virtual reality is a real or simulated environment in which a perceiver experiences a sense of being in that environment by means of a communication medium.}
\end{displayquote}

The connection between VR hardware such as the HTC Vive, the Oculus Rift or the PS VR and this concept of virtual reality is, of course then, that these devices amplify user's sense of being in virtual environments.

\subsection{Immersion}
\label{subsec:immersion}

As mentioned before, a term used frequently when talking about the VR medium is immersion. Slater defines immersion as a property useful for the comparison of different virtual reality systems. He borrows the concept of sensorimotor contingencies (SC) from behavioural and brain scientists O'Regan and Noë. SCs are the actions that we know we can perform in order to control our perception: bending down and turning our heads in order to see underneath something and further defines Valid Actions of a virtual reality system as the actions that the user can take that result in changes in perception (sensorimotor actions), or changes to the environment (effectual actions) \parencite{Slater2009}. By Slater's definition:

\begin{displayquote}
\textit{Immersion is a property of the Valid Actions that are possible within a system.}
\end{displayquote}

\subsection{Place Illusion and Plausibility Illusion \textit{in lieu of} Presence}
\label{subsec:PIandPsi}

The fact that the medium we are dealing with is defined in terms of the concept of presence should make the importance of that concept in this context evident. The concept of presence is however not one that fully understood. There are many theories about presence and as Schuemie et al. point out, even without being in contradiction with each other, their differences can have different implications. An overview of the existing research on presence in virtual reality can be found in their survey paper \parencite{Schuemie2001}.

Slater's definition of immersion focuses on the affordances \parencite{Norman} of virtual reality systems (hardware and software combined) and he builds on it to analyze the concept of presence (in Steuer's terms, telepresence), now focusing on the user experience. He proposes using what he calls Place Illusion (PI) and Plausibility Illusion (Psi) as factors to explain presence \parencite{Slater2009}:

\begin{displayquote}
\textit{Place Illusion is the illusion of being in a place in spite of the sure knowledge that you are not there. PI is related mainly to the physics of virtual situations. "It is maintained through synchronous correlations between the act of moving an concomitant changes in the images that form perception" Slater explains.}
\end{displayquote}

\begin{displayquote}
\textit{Plausibility Illusion is the illusion that what is apparently happening is really happening in spite of the knowledge that it is not. Psi is determined by the extent to which the system can produce events that directly relate to the user, and the overall credibility of the scenario being depicted in comparison with expectations. It is the automatic and rapid response of the user to the question: is this really happening?}
\end{displayquote}

According to Slater, PI and Psi fuse when we consider the body. A virtual body that is in principle external and disconnected from us is first perceived to be in the space where we sense ourselves to be and then discovered to respond to our own body's movements with analogous and synchronous movements.

The convergence of PI and Psi in the user's body seems to place it as a key point in the topic of presence. \parencite{Schubert1999} analyze presence, understood as the sense of being in a virtual environment, from the perspective of embodied cognition. They propose an interpretation of presence as embodied presence: "Presence develops from the representation of navigation (movement) of the own body (or body parts) as a possible action in the virtual world". They use Glenberg's theories explaining that we understand the environment in terms of patterns of possible action, and argue that part of the process of developing presence is constructing a mental model of the possible actions within the virtual space and "the more the mediated stimuli follow embodied constraints (e.g., coupling with body movement), the easier [the construction is]".

\subsection{Sense of Embodiment}
\label{subsec:embodiment}

In dealing with virtual hands, we are considering what connection, if at all, one might experience towards a virtual body. \parencite{Kilteni2012} propose the notion of Sense of Embodiment (SoE) to address the issue of "whether it is possible to experience the same sensations towards a virtual body as toward a biological body":

\begin{displayquote}
\textit{SoE toward a body B is the sense that emerges when B's properties are processed as if they were the properties of one's own biological body.}
\end{displayquote}

Kilteni et al. further explain SoE in terms of three subcomponents:

\begin{displayquote}
\begin{itemize}
\item \textit{Sense of Self-Location: one's spacial experience of being inside a body.}
\item \textit{Sense of Agency: "the sense of having global motor control, including the subjective experience of action, control, intention, motor selection and the consious experience of will"} (Blanke and Metzinger's definition, as cited in \parencite{Kilteni2012})\textit{, resulting from the comparison between the predicted sensory consequences of one's actions and the actual sensory consequences.}
\item \textit{Sense of Body Ownership: one's self attribution of the body, implying the sense that the body is the source of the experienced sensations.}
\end{itemize}
\end{displayquote}

\section{TODO}

\begin{itemize}
\item embodied presence \parencite{Schubert1999}
\item reference presence survey \parencite{Schuemie2001}
\item Define virtual reality, presence, immersion and sense of embodiment.
\item Map terms to proposed desirable properties of virtual hands.
\item Survey alternative definitions
\item Fidelity coherence and expectations \parencite{Nowak2003, Argelaguet2016}
\item Why do we care about presence and SoE? -> distinctive quality of the medium, embodied cognition
\item Address Norman's critique of movement based interaction \parencite{Gillies2016}
\item Rubber hand illusion and adjusted hand control model \parencite{Sanchez-Vives2010}, \parencite{Fourneret1998}
\item presence / place illusion / SoE with different sensorimotor contingencies or not in the same capacity as we are present in the real world
\end{itemize}