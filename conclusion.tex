This work has attempted to glean knowledge about how to control hands in Virtual Reality using positional tracking devices. Specifically, it has focused on the issue of how to deal with the conflicts that arise between the position and pose of the user's real hands and the physical constraints that the user might expect the virtual environment they are interacting with to impose on their virtual hands: we have referred to this issue as the Virtual Constraints Problem. Our proposed approach to deal with this problem is to adjust the virtual hands position, rotation and finger pose, deviating from the tracking data and other user input in a way that respects the intentions of the user in order to minimize the loss  of the sense of "presence" as explained in terms of Place Illusion, Plausibility Illusion and Sense of Embodiment, discussed in Chapter \ref{chap:theory}. We referred to ways of controlling hands that work like this as Adjusted Hand Control Models.

The controlled experiments conducted in order to compare an example Adjusted Hand Control Model with an Unadjusted Hand Control Model based on the game Job Simulator showed positive results and, together with the recount of prototyping experiments presented in Chapter \ref{chap:experimentalAnalysis}, could provide guidance as to how to proceed in the future when developing Adjusted Hand Control Models. The test setup was however not perfect and we believe there were confounding factors that might have affected the results. Conducting new experiments after solving some of the discovered issues would produce more reliable results. Alternatively, experiments could be performed following established evaluation techniques developed by other researchers for the purpose of measuring "presence" (see overview in \parencite{Schuemie2001}).

Furthermore, we find that previous research could support the hypothesis that separating the virtual hands from the real body as Adjusted Hand Control Models require might not have a negative impact on "presence". At the same time, other developers have begun to create VR games like Wilson's Heart and Lone Echo that use Adjusted Hand Control Models or at any rate approaches that also separate parts of the virtual body from the real body under certain circumstances. The results shown by these developers are promising at the very least.


\section{Future Work}
\label{sec:futureWork}

\begin{itemize}
\item Testing with established presence evaluation methods.
\item Grip altering position of hand instead of only fingers
\item Rotation improvements and stability. Non-kinematic rigidbody shortcut gives us things for free, but doesn't allow us enough control. Serves as proof of concept.
\item Methodology reflection?
\end{itemize}