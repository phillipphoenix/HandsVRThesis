\begin{itemize}
\item Introduction for the implementation section.
\item Introduction should perhaps introduce the topics of this chapter.
\end{itemize}

This chapter delves into the categorization of approaches that can be taken when implementing hands in VR and which of these were used for the hands implemented during the thesis project.

\todo{How much should be explained here about the general approach ie adjust input?}

\section{Categorization of approaches}
\label{sec:categorizationOfApproaches}
\begin{itemize}
\item Lead into the explanation for the filter variables.
\item Display filter variable table.
\item Describe several examples of filter variable combinations and show image sequences.
\item (Describe the reasons and effects for filtering on the different variables)?
\end{itemize}

Hands can be implemented in different ways which will lead to different ways of interacting with the world and a different feel for the player. One way of approaching how to implement hands is to filter the player's input. 3 filtering variables will be described below: position filtering, rotation filtering and finger position fingering. These three filtering variables all filter different player input in order to control the hand in different ways.

\missingfigure[figwidth=15cm, figheight=6cm]{Table: Filter variable combinations}

\subsection{Position filtering}
\label{subsec:categoryPositionFiltering}
\textbf{Questions to answer in this section:}
\begin{itemize}
\item What does it mean to filter the player's positional input?
\item Why do we want to filter the player's positional input?
\end{itemize}


\textbf{What does it mean to filter the player's positional input?}\\
Filtering on the player's positional input means that in certain cases the hand will deviate from the controller's position.

\textbf{Why do we want to filter the player's positional input?}\\
\todo{Careful here. Hypothesis is a better word here and I am hesitant to declare it our own. Not embodiment, Sense of Embodiment. Might abreviate to SoE.} One of the assumptions put forward in this paper is that it's possible to increase the player's embodiment by simulating more realistic behaviour for hands around objects in the world. One of the first steps here is to not allow the hands to penetrate objects. By using position filtering to deviate the hand from the controller position when trying to penetrate an object, the hand can interact more realistically with the world.

The following two figures show an image sequence of a hand with no input filtering and an image sequence with a hand that uses position filtering. The second figure shows that the hand is stopping when it reaches the object, whereas in the first figure the hand moves through the object.

\missingfigure[figwidth=15cm]{Image sequence: No filtering}
\missingfigure[figwidth=15cm]{Image sequence: Position filtering}

\subsection{Rotation filtering}
\label{subsec:categoryRotationFiltering}
\textbf{Questions to answer in this section:}
\todo{Alternative term: orientation}
\begin{itemize}
\item What does it mean to filter the player's rotational input?
\item Why do we want to filter the player's rotational input?
\begin{itemize}
\item Reduce position filtering.
\item Display player's intentions.
\todo{Diagetically communicate available interactions.}
\todo{Allows for more detailed hand animation depending on available interactions. Control?}
\end{itemize}
\end{itemize}

\textbf{What does it mean to filter the player's rotational input?}\\
\todo{Adjusting might be a better word?} Filtering on the rotation input means to rotate the hand differently in the virtual world compared to the rotation of the controller.

\textbf{Why do we want to filter the player's rotational input?}\\
\todo{Mention stiffness with only position filtering}
Rotation filtering can be used to display what is assumed to be the player's intention when they interact with the world. One example could be the intention of the player when they approach a wall with their hand. In this case a reasonable assumption would be that the player's intention is to rotate the hand so that the palm faces the wall (See image sequences below). Besides being useful when wanting to adapt the hand to the player's intentions, rotation filtering can also be used in order to reduce the amount of position filtering needed. This means that rotating the hand will allow the distance between the hand and the controller due to position adjusting to be reduced (See image sequence below).

\todo{use refs}
The first of the two figures below shows an image sequence of a hand using only rotation filtering and the second figure shows an image sequence of a hand using both position and rotation filtering where the rotation filtering is implemented as palm-first towards the surface.

\missingfigure[figwidth=15cm]{Image sequence: Rotation filtering}
\missingfigure[figwidth=15cm]{Image sequence: Position and rotation filtering}

\subsection{Finger position filtering}
\label{subsec:categoryFingerFiltering}
\textbf{Questions to answer in this section:}
\begin{itemize}
\item What does it mean to filter the player's finger position input?
\item Why do we want to filter the player's finger position input?
\begin{itemize}
\item Reduce position filtering.
\item Display player's intention.
\end{itemize}
\end{itemize}

\textbf{What does it mean to filter the player's finger position input?}\\
\todo{Maybe the whole control scheme thing shouldn't be mentioned here, but earlier as a general premise: hand position and rotation taken from tracking, finger positions from trigger.}
\todo{With the included/default. There are plenty of commercially available gloves, which still would require us to filter input like with the hand position and rotation. What does "direct mapping" mean?}With the current commercially available controllers for the HTC Vive (and Oculus Rift) the control of individual fingers is limited. There is no tracking of the fingers and there isn't a direct mapping between the buttons available and the control of the fingers. Making the fingers adapt to their environment is therefore the task of the filtering system. The filtering system can bend the fingers individually, placing the finger tip at different locations relative to the rest of the hand.

\textbf{Why do we want to filter the player's finger position input?}\\
Here, like with the rotation filtering, a certain amount of assumptions have to be made about what the player's intend is. When a player's hand is approaching an object that can be grabbed, the most common case might be that they are trying to grab the object. If this is the case, adjusting the fingers to form a grip could be a natural behaviour.
\todo{rewritten}

\missingfigure[figwidth=15cm]{Image sequence: Position and finger position filtering}

\section{Description of how we evaluated hand iterations ...}
\label{sec:DESCRIPTIONOFEVALUATIONSCENARIOS}
\todo{should this be methodology?}
\section{Implementing filtering}
\label{sec:implementingFiltering}
Something about how the implementation of each type of filtering can be done in several ways and perhaps how the order of the types used in a combination might create a different feel.

\subsection{Position filtering}
\label{subsec:implementationPositionFiltering}
\textbf{Questions to answer in this section:}
\begin{itemize}
\item How can we filter the player's positional input?
\begin{itemize}
\item Depenetration.
\item Physics system.
\todo{Restriction}
\end{itemize}
\end{itemize}

\textbf{How can we filter the player's positional input?}\\
Position filtering in the case of disallowing the penetration of objects can be implemented in several ways. One of the methods used during this thesis project is to detect a collision between the hand and an object in the world and take action when such a collision occurs. The manual collision checking can be implemented by checking for nearby object colliders and using ray casts or sweeps to anticipate the collision.
Another implementation uses the physics system to handle collisions between the hands and obstacles in the world. The hands are moved by setting their velocities in such a way that they would reach their target within the current frame.

\subsection{Rotation filtering}
\label{subsec:implementationRotationFiltering}
\textbf{Questions to answer in this section:}
\begin{itemize}
\item How can we filter the player's rotational input?
\begin{itemize}
\item Manual rotation filtering when approaching obstacles.
\item Differentiation between object types and angles of approach.
\item Physics system.
\end{itemize}
\end{itemize}

\textbf{How can we filter the player's rotational input?}\\
Compared to position filtering, the way to implement rotation filtering is very much influenced by the assumptions about the player's intentions. When a player approaches a wall with their hand one assumption could be that the player's intention is to let their hand face the wall palm-first. This would manifest itself as a form of rotation filtering where the hand will rotate when nearing walls. While a player might want to approach a wall palm-first, their intentions might be different when approaching other types of objects like certain smaller grabbable objects which could have a specific pose for the hand to rotate towards. Another parameter which might help decide how to adapt the rotation is the angle of approach. When approaching a surface the rotation target might differ depending on if it's the front or the back of the hand that is facing the surface.

\todo{"pressed firmly"?}
Another approach entirely is to let the physics system handle rotation filtering. This at first will not make a difference when the hand is touching an object, but when pressed firmly against a surface the hand will rotate as to reduce the distance between itself and the controller position.

\subsection{Finger position filtering}
\label{subsec:implementationFingerFiltering}
\begin{itemize}
\item How can we filter the player's finger position input?
\begin{itemize}
\item Filtering to avoid obstacles.
\item Filtering to anticipate player intent.
\item Differentiation between object types and angles of approach.
\end{itemize}
\end{itemize}

\textbf{How can we filter the player's finger position input?}\\
Implementations of finger position filtering include using an animation system to animate the fingers together or individually to different poses and the use of an Inverse Kinematics (IK) system to infer finger pose from finger tip position and hand position and orientation.

\todo{Obviously you skipped the actual methods?}

\section{Hands and their filterings}
\label{sec:LABELABOUTHANDSVERSIONS}
\begin{itemize}
\item Describe each hand and what filtering variables they use.
\item Relate hands to the above filtering variable descriptions.
\end{itemize}

\section{Miscellaneous shitz}
\label{sec:MISCELLANEOUSSHITZ}

\subsection{Hand visualization}
\label{subsec:handVisualization}

\subsection{Rumblez!}
\label{subsec:RUMLBEZ}

\subsection{Grabbing system details}
\label{subsec:grabbingSystem}