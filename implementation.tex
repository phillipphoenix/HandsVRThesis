\begin{itemize}
\item Introduction for the implementation section.
\item Introduction should perhaps introduce the topics of this chapter.
\end{itemize}

\textbf{What is the implementation chapter about?}\\
This chapter delves into the implementation of several versions of hands for VR. These different versions have similarities and differences in their approach to the problem. In the following sections we will present a categorization of approaches that can be taken when implementing hands in VR, what effect these different approaches have and their pros and cons.

\textbf{What stuff did we use during the project?} (headset and control scheme, for instance)\\
Since we have been using the HTC Vive and its default controllers as input to control the hands our approaches are more applicable to hardware that has the same accordances as these controllers. We use the position and orientation input from the controllers to position and rotate the hands in the virtual space and we use the trigger button to indicate how much the fingers should bend or how closed the hand should be.

\textbf{How does this lead into the categorizations?}\\
These three inputs are the base of the categorizations of approaches. Each of these inputs can be filtered, by which is meant that the input can be modified or skipped. Filtering on one of the inputs can be seen as using that input as the parameter for a function, which returns a new result. A hand can have a filter for either none of the inputs or one of the inputs or more. Different filtering combinations will give a different behaviour for the hand in the virtual world and might result in the hand seeming more realistic when interacting with its environment.

\section{Categorization of approaches}
\label{sec:categorizationOfApproaches}
\begin{itemize}
\item Lead into the explanation for the filter variables.
\item Display filter variable table.
\item Describe several examples of filter variable combinations and show image sequences.
\item (Describe the reasons and effects for filtering on the different variables)?
\end{itemize}

The three player inputs mentioned above (position, orientation and how closed the hand should be) are the base of the categorization of approaches. Each of these inputs can be filtered, by which is meant that the input can be modified or skipped. Filtering on one of the inputs can be seen as using the input as the parameter for a function, which returns a new result which will be used instead of that input. A hand can be implemented with filters on several inputs at once and can have different filters depending on the context. Different filterer combinations will give a different behaviour for the hand in the virtual world and can result in the hand seeming more realistic when interacting with its environment.

\missingfigure[figwidth=15cm, figheight=6cm]{Table: Filter variable combinations}

\subsection{Position filtering}
\label{subsec:categoryPositionFiltering}
\textbf{Questions to answer in this section:}
\begin{itemize}
\item What does it mean to filter the player's positional input?
\item Why do we want to filter the player's positional input?
\end{itemize}

\textbf{What does it mean to filter the player's positional input?}\\
Filtering on the player's positional input means that in certain cases the hand will deviate from the controller's position.

\textbf{Why do we want to filter the player's positional input?}\\
One of our hypotheses is that it's possible to increase the player's sense of embodiment by simulating more realistic behaviour for hands around objects in the world. One of the first steps here is to not allow the hands to penetrate objects. By using position filtering to deviate the hand from the controller position when trying to penetrate an object, the hand can interact more realistically with the world.

The following two figures show an image sequence of a hand with no input filtering and an image sequence with a hand that uses position filtering. The second figure shows that the hand is stopping when it reaches the object, whereas in the first figure the hand moves through the object.

\missingfigure[figwidth=15cm]{Image sequence: No filtering}
\missingfigure[figwidth=15cm]{Image sequence: Position filtering}

\subsection{Rotation filtering}
\label{subsec:categoryRotationFiltering}
\textbf{Questions to answer in this section:}
\todo{Alternative term: orientation}
\begin{itemize}
\item What does it mean to filter the player's rotational input?
\item Why do we want to filter the player's rotational input?
\begin{itemize}
\item Reduce position filtering.
\item Display player's intentions.
\item Diagetically communicate available interactions.
\item Allows for more detailed hand animation depending on available interactions. Control?
\item Mention stiffness with only position filtering
\end{itemize}
\end{itemize}

\textbf{What does it mean to filter the player's rotational input?}\\
To use a rotation filter is to deviate the hands rotation from the current orientation of the controller. In certain contexts it can be beneficial filter on rotation on order for the hand to seem more alive and realistic. Hands can adapt their rotation in several ways and different approaches might be taken depending on the context. When approaching a wall with their hand a player's intention in the real world might be to place their hand flat on the wall (palm-first). This behaviour can be implemented using a rotation filter, which takes effect when the hand is approaching a surface.


\todo[inline]{Adjusting might be a better word?} Filtering on the rotation input means to rotate the hand differently in the virtual world compared to the rotation of the controller.

\textbf{Why do we want to filter the player's rotational input?}\\
Rotation filtering can be used to display what is assumed to be the player's intention when they interact with the world. One example could be the intention of the player when they approach a wall with their hand. In this case a reasonable assumption would be that the player's intention is to rotate the hand so that the palm faces the wall (See image sequences below). Besides being useful when wanting to adapt the hand to the player's intentions, rotation filtering can also be used in order to reduce the amount of position filtering needed. This means that rotating the hand will allow the distance between the hand and the controller due to position adjusting to be reduced (See image sequence below).

\todo[inline]{use refs}
The first of the two figures below shows an image sequence of a hand using only rotation filtering and the second figure shows an image sequence of a hand using both position and rotation filtering where the rotation filtering is implemented as palm-first towards the surface.

\missingfigure[figwidth=15cm]{Image sequence: Rotation filtering}
\missingfigure[figwidth=15cm]{Image sequence: Position and rotation filtering}

\subsection{Finger position filtering}
\label{subsec:categoryFingerFiltering}
\textbf{Questions to answer in this section:}
\begin{itemize}
\item What does it mean to filter the player's finger position input?
\item Why do we want to filter the player's finger position input?
\begin{itemize}
\item Reduce position filtering.
\item Display player's intention.
\end{itemize}
\end{itemize}

\textbf{What does it mean to filter the player's finger position input?}\\
Filtering the finger positions is about placing the finger tips in space relatively to the rest of the hand or put differently; stretching and bending the fingers. The fingers can be filtered as a group or individually and different contexts can determine different filters.

\textbf{Why do we want to filter the player's finger position input?}\\
Here, like with the rotation filtering, a certain amount of assumptions have to be made about what the player's intend is. When a player's hand is approaching an object that can be grabbed, the most common case might be that they are trying to grab the object. If this is the case, adjusting the fingers to form a grip could be a natural behaviour.

\missingfigure[figwidth=15cm]{Image sequence: Position and finger position filtering}

\section{Description of how we evaluated hand iterations ...}
\label{sec:DESCRIPTIONOFEVALUATIONSCENARIOS}
\todo[inline]{should this be methodology?}

\section{Implementing filtering}
\label{sec:implementingFiltering}
Something about how the implementation of each type of filtering can be done in several ways and perhaps how the order of the types used in a combination might create a different feel.

\subsection{Position filtering}
\label{subsec:implementationPositionFiltering}
\textbf{Questions to answer in this section:}
\begin{itemize}
\item How can we filter the player's positional input?
\begin{itemize}
\item Depenetration.
\item Physics system.
\item Restrictions by using the different filter methods.
\end{itemize}
\end{itemize}

\textbf{How can we filter the player's positional input?}\\
Position filtering in the case of disallowing the penetration of objects can be implemented in several ways. One of the methods used during this thesis project is to detect a collision between the hand and an object in the world and take action when such a collision occurs. The manual collision checking can be implemented by checking for nearby object colliders and using ray casts or sweeps to anticipate the collision.
Another implementation uses the physics system to handle collisions between the hands and obstacles in the world. The hands are moved by setting their velocities in such a way that they would reach their target within the current frame.

\subsection{Rotation filtering}
\label{subsec:implementationRotationFiltering}
\textbf{Questions to answer in this section:}
\begin{itemize}
\item How can we filter the player's rotational input?
\begin{itemize}
\item Manual rotation filtering when approaching obstacles.
\item Differentiation between object types and angles of approach.
\item Physics system.
\end{itemize}
\end{itemize}

\textbf{How can we filter the player's rotational input?}\\
Compared to position filtering, the way to implement rotation filtering is very much influenced by the assumptions about the player's intentions. When a player approaches a wall with their hand one assumption could be that the player's intention is to let their hand face the wall palm-first. This would manifest itself as a form of rotation filtering where the hand will rotate when nearing walls. While a player might want to approach a wall palm-first, their intentions might be different when approaching other types of objects like certain smaller grabbable objects which could have a specific pose for the hand to rotate towards. Another parameter which might help decide how to adapt the rotation is the angle of approach. When approaching a surface the rotation target might differ depending on if it's the front or the back of the hand that is facing the surface.

\todo{"pressed firmly"?}
Another approach entirely is to let the physics system handle rotation filtering. This at first will not make a difference when the hand is touching an object, but when pressed firmly against a surface the hand will rotate as to reduce the distance between itself and the controller position.

\subsection{Finger position filtering}
\label{subsec:implementationFingerFiltering}
\begin{itemize}
\item How can we filter the player's finger position input?
\begin{itemize}
\item Filtering to avoid obstacles.
\item Filtering to anticipate player intent.
\item Differentiation between object types and angles of approach.
\end{itemize}
\end{itemize}

\textbf{How can we filter the player's finger position input?}\\
Implementations of finger position filtering include using an animation system to animate the fingers together or individually to different poses and the use of an Inverse Kinematics (IK) system to infer finger pose from finger tip position and hand position and orientation.

\todo{Obviously you skipped the actual methods?}

\section{Hands and their filterings}
\label{sec:LABELABOUTHANDSVERSIONS}
\begin{itemize}
\item Describe each hand and what filtering variables they use.
\item Relate hands to the above filtering variable descriptions.
\end{itemize}

\section{Miscellaneous shitz}
\label{sec:MISCELLANEOUSSHITZ}

\subsection{Hand visualization}
\label{subsec:handVisualization}

\subsection{Rumblez!}
\label{subsec:RUMLBEZ}

\subsection{Grabbing system details}
\label{subsec:grabbingSystem}