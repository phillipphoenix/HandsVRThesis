\begin{quotation}
\textit{We are trying to learn thinking. Perhaps thinking, too, is just something like building a cabinet. At any rate, it is a craft, a "handicraft." "Craft" literally means the strength and skill in our hands. The hand is a peculiar thing. In the common view, the hand is part of our bodily organism.}

\textit{But the hand's essence can never be determined, or explained, by its being an organ which can grasp. Apes, too, have organs that can grasp, but they do not have hands. The hand is infinitely different from all grasping organs - paws, claws, or fangs - different by an abyss of essence. Only a being who can speak, that is, think, can have hands and can be handy in achieving works of handicraft. }

\textit{But the craft of the hand is richer than we commonly imagine. The hand does not only grasp and catch, or push and pull. The hand reaches and extends, receives and welcomes - and not just things: the hand extends itself, and receives its own welcome in the hands of others. The hand holds. The hand carries. The hand designs and signs, presumably because man is a sign. hands fold into one, a gesture to carry man into the great oneness. The \todo{Is this correct? Isn't there a word missing (like hand)?}is all this, and this is the true handicraft. Everything is rooted here that is commonly known as handicraft, and commonly we go no further. But the hand's gestures run everywhere through language, in their most perfect purity precisely when man speaks by being silent. And only when man speaks, does he think not the other way around, as metaphysics still believes. Every motion of the hand in every one of its works carries itself through the element of thinking, every bearing of the hand bears itself in that element. All the work of the hand is rooted in thinking. Therefore, thinking itself is man's simplest, and for that reason, hardest handiwork, if it would be accomplished at its proper time.}
\\

\begin{flushright}
- Martin Heidegger, "\textit{What is called thinking?}" \footcite{Heidegger1968}
\end{flushright}
\end{quotation}