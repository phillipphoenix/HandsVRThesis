\section{Results}
\label{sec:results}

The information gathered in the user tests through the questionnaire and the interviews is presented in this section. The full questionnaire and a link to the raw data can be found in Annex \ref{REF HERE} The questionnaire answers are divided in two graphs: Figure \ref{fig:abGraph} summarizes the results of the A/B test-type questions, while Figure \ref{fig:descriptionGraph} contains the results of the questions where the testers where asked to select descriptions from a list to describe each hand version (they were also given the option to write their own description). Furthermore, Figure \ref{fig:combinedDescriptionsGraph} summarizes the information from the description selection questions by combining the number testers that selected opposing terms into one term for each hand version.

\todo[inline]{Reference to questionnaire + interview + raw data link Annex}

Figure \ref{fig:interviewResults} presents the the result of performing a qualitative data analysis of the interview notes. The notes were coded and categorized following an inductive process, meaning that they were created based on the themes of the interview notes \parencite{Burnard2008}, and the number of occurrences of the labels in the notes was counted as a measure of the relevance of each label.

We conducted the test procedure with 10 graduate-level university students (7 men and 3 women). It should be noted that all of the test subjects part of the games program at IT University of Copenhagen, so these results might not be fully representative of the general VR game player/user population.

\begin{figure}[h]
\centering
\includegraphics[width=0.8\textwidth]{abGraph.png}
\caption{Graph showing the number of testers that preferred each version of the hands in each of the aspects described by the A/B test-type questions: "With which version did you prefer petting/stroking the creatures?" (Stroking), "With which version did you prefer hitting things?" (Hitting), "With which version did you prefer pushing things?" (Pushing), "With which version did you prefer grabbing things?", (Grabbing), "Which version did you prefer in terms of how they interacted with objects that you could move?" (Movable), "Which version did you prefer in terms of how they interacted with objects that you could NOT move?" (Immovable).}
\label{fig:abGraph}
\end{figure}

The labels in Figure \ref{fig:interviewResults} have the following meanings. In the \textit{Fingers} category, \textit{Stroking Good} means that the tester expressed that the finger adjustment improved the stroking experience: \textit{Hover Good} means that they liked the finger adjustment or hover around movable objects in a general sense. \textit{Impossible Grip} means that they expressed a dislike for the cases when the hands grabbed objects in a way that would be impossible in real life because it would not be possible to get a grip e.g. on a completely flat surface. \textit{Loose Grip} is similar to the previous one, and refers to disliking the cases when they grabbed an object and the hand pose didn't make contact with the object. \textit{Hidden Grab} refers to users expressing a dislike for the Standard Hands disappearing when an object is grabbed. \textit{Flat Slap} refers to user's pointing out that because of the finger adjustment or hover around movable objects, hitting the objects with a flat hand wasn't possible. \textit{Unnoticed} refers to the cases when the users didn't notice the finger adjustment in the Adaptive Hand. \textit{Inconsistent} means that the tester mentioned the fact that the fingers didn't adjust around immovable objects and were inconsistent in that sense.


In the \textit{Rumble} category, the label \textit{Unpleasant} means that testers expressed that the haptic feedback or rumble given by the controllers was unpleasant to them. \textit{Satisfying} means that the users mentioned that the rumble was satisfying at least in some of the cases when they felt it. \textit{Informative} means that the testers found the rumble informed them of about the virtual environment, virtual events or the environment constraints or "rules". \textit{Innacurate} means that the testers described the rumble as "innacurate", "imprecise" or "unpolished". \textit{Intimate} means that the user considered that the rumble made the touch interaction feel more intimate. \textit{Confusing} means that the user expressed feeling confused about the why the rumble was being produced.

\begin{figure}[H]
\centering
\includegraphics[width=\textwidth]{descriptionGraph2.png}
\caption{Graph of the number of testers that chose each of the words in a list to describe the two versions of the hand that were shown. A few of the testers added descriptions of their own, which are quoted under the graph.}
\label{fig:descriptionGraph}
\end{figure}

\begin{figure}[H]
\centering
\includegraphics[width=\textwidth]{combinedDescriptionsGraph.png}
\caption{Graph the synthesizes the results from the word selection questions (shown in Figure \ref{fig:descriptionGraph}) by combining opposite terms into one term. For each hand, the new terms are calculated as follows: Responsiveness = Responsive - Unresponsive; Pleasure = Pleasant - Unpleasant; Controllability = Controllable - Uncontrollable; Consistency = Consistent - Inconsistent; Intuitiveness = Intuitive - Confusing; Precision = Precise - Imprecise.}
\label{fig:combinedDescriptionsGraph}
\end{figure}

In the \textit{Physics} category, \textit{Natural Grab} means that users liked and found it easier with the Adaptive Hands to lift objects and manipulate purely using the physics system (e.g. lifting an object by placing their hands on opposite side of the object and moving their hands towards each other) instead of making use of the "artificial" trigger grab mechanic. \textit{Penetration} means that the testers expressed liking the fact that their hands could not move into immovable objects. \textit{Weight} means that the users mentioned that objects seemed to have weight when interacting with them with the Adaptive Hands. \textit{Hard Lifting} refers to users expressing that the Natural Grab was better, but not easy enough with the Adaptive Hands. \textit{Unnecessary} means that the user expressed not feeling inclined to break the physics rules of the environment, because there were contextual signifiers \parencite{Norman2010} that made each virtual object's affordances clear.

In the \textit{Communication} category, \textit{Wireframe Late} means that the users found that the wireframe version of the Adaptive Hand shown when the hand was separated from the controller appeared too late (for example because the controller is inside of a virtual object). \textit{Wireframe Good} means that they found that the wireframe visualization clarified that the system was working as intended and that there was no "bug" when the virtual hands separated from the controller. \textit{No Grab Feedback} means that a user expressed missing some kind of feedback when in the moment when the grab action is completed and the grabbed object is attached to the grabbing hand. \textit{Sound Feedback} means that the user expressed liking the sound effects produced by certain interactions. \textit{Feedback Good} refers to a general positive appreciation of the feedback with the Adaptive Hands.

\begin{figure}[H]
\centering
\includegraphics[width=\textwidth]{interviewResults.png}
\caption{Each table shows a category and its labels with the number of mentions found in the interview notes.}
\label{fig:interviewResults}
\end{figure}

Under \textit{Stability}, \textit{Shaking} means that the users noticed and expressed dislike for the cases when the Adaptive Hand shook or moved in a "jumpy" fashion when interacting with objects. \textit{Discontinuity} refers to users noticing small discontinuities in the finger movement in some situations.

In the \textit{Hand Adjustment} category, \textit{Bothered} means that the users expressly disliked when the Adaptive Hands separated from the controllers in terms of position or rotation (not including finger adjustments). \textit{Not Bothered} means that the users expressly manifested not being bothered or in some cases not noticing when the hands separated from the controller in the previous sense.

Finally, under \textit{General}, \textit{Body Illusion} refers to users expressing that the sometimes found themselves behaving as if they had parts their body that the virtual body didn't have. \textit{No Preference} means that the users expressly recognized that there was "nothing wrong" with the Standard Hand or that they had no strong feelings of preference towards either version. \textit{Adaptive Real} refers to users that mentioned that the Adaptive Hands felt "more real" to them.

\subsection{Discussion}
\label{subsec:discussion}

The results from the A/B questions on their own seem to be pretty conclusive in favor of the Adaptive Hands, but at the same time they provide very little detail. However, looking at the results from the description questions and specially the interviews, reveals that the issue is far more intricate than the A/B results indicate.

Reviewing Figure \ref{fig:abGraph}, only two of the questions were slightly contentious. The lower preference in favor of the Adaptive Hands in \textit{Movable} might be due to the fact that the Standard Hands already work reasonably with objects that can be moved in most cases because these objects can be forced away from the hands, unlike in the case of immovable objects, where the Standard Hands just go through the objects.

The same argument might partly explain why \textit{Hitting} was even more divided. Hitting immovable objects with the Standard Hands is not really possible, so one would expect \textit{Hitting} to be more clearly in favor of the Adaptive Hands. However, we observed that players didn't interact with immovable objects with their hands often and some of the users even reported in the interview that they did not feel inclined to interact with the elements of the virtual environment that seemed to not have a specific interaction (\textit{Unnecessary}), which in the case of Touch Island happen to be the immovable objects. An explanation then could be that the testers simply didn't experience the difference between the Standard Hands and the Adaptive Hands in terms of hitting immovable objects. Reflecting on the design of the the test environments, particularly Touchy Island, giving the immovable objects interactive properties might have naturally encouraged testers to explore this aspect or the hands more.

\todo[inline]{Argue why Hitting was low but not Immovable?}

\todo[inline]{Cherries and trees example?}

\todo[inline]{Cover the questions that are in favor of the Adaptive Hands?}

The comparison becomes less straightforward looking at the rest of the data: Figure \ref{fig:combinedDescriptionsGraph}, interestingly, places the Standard Hands ahead of the Adaptive Hands in most aspects. The only aspect in which the Adaptive Hands seem to stand out is \textit{Resposiveness}. Understood as describing whether the hands "react" to the environment, this seems like a reasonable description for the Adaptive Hands since they both don't penetrate objects and their fingers react in a more dynamic way to movable objects near them. This meaning of responsiveness in the context of hands for VR is, however, not necessarily a desirable quality, since it comes at the cost of behavior that is only indirectly caused by the user's movements, which could translate in losses in Sense of Agency. On the other hand, it could enhance the tactility of the interactions with virtual objects and could be used to diegetically communicate that certain actions are available with objects. The Standard Hands also use automatic finger movements to indicate that actions are available with the object the hand is hovering over, which should be at least as objectionably in terms of Sense of Embodiment, which makes us think that \textit{Responsiveness} was used by the users in a positive sense.

What might seem like a simple tie in \textit{Precision} and a close score in \textit{Controllability}, we would argue could be in fact a very positive sign for the Adaptive Hands and could clarify the doubts with respect to the value of \textit{Responsiveness}. Given that the Adaptive Hands deviate from the user input, the fact that they are considered equally precise and almost equally controllable by the testers can mean that they do not perceive a loss of control when using the Adaptive Hands, that is, there doesn't seem to be a loss of Sense of Agency after all.

The biggest difference in favor of the Standard Hands is in \textit{Consistency}. We attribute this to tow factors: firstly, the finger adjustment in the Adaptive Hands only happens around movable objects. This is also the case with the Standard Hands, but if the automatic finger movement in the Standard Hands is only interpreted as a signifier of the possibility of grabbing an object whereas in the Adaptive Hand finger adjustment is thought of as a natural behavior around and physical object, more related with feeling the objects with the fingers. A more straightforward suspect for this big difference in \textit{Consistency} is the stability issues present in the Adaptive Hands, which were noted by several of the users in the interviews.

By first discussing the interview categories \textit{Rumble} and \textit{Communication}, explaining the results for the descriptions \textit{Pleasure} and \textit{Intuitiveness} becomes easier. The most frequent opinion about the rumble in the Adaptive Hands is that it was "unpleasant", "unnatural" and "uncomfortable" (reported also by one of the testers in the questionnaire). This aspect of the hands turns out to be a rather contentious one, as it seems that some users did find that the rumble was satisfying in certain interactions like hitting and pushing objects or even "intimate" when stroking or caressing the creature in Touchy Island. It was also described as "confusing", "inaccurate", "imprecise" and "unpolished", which might explain the division of opinions. We consider this to have become a confounding factor for the evaluation of the adjusted hand control model, since only the Adaptive Hands provided haptic feedback. Given these explanations, it seems likely that the differences in \textit{Pleasure} are linked to the problems with the rumble. This seems all the more likely when reviewing the original Figure \ref{fig:descriptionGraph} and finding that the Adjusted hands actually were described as "pleasant" by more users than the Standard Hands, but that fact was neutralized by other users describing them as "unpleasant". Our conclusion is then that other aspects of the Adaptive Hands were in fact very pleasant to the testers and had there not been any haptic feedback or if it had been working differently, the results could be far better with the Adaptive Hands.

The difference in \textit{Intuitiveness} might in fact be also been tainted by the negative experience that the rumble caused in the users. Testers described the Adaptive Hands as "confusing" in the questionnaire (remember that the \textit{Intuitiveness} score is negatively affected by users describing the hands as "confusing"), but also explained in the interviews that the rumble was confusing. Testers also seemed to understand how the Adaptive Hand worked better when they saw wireframe visualization, but some of the users indicated that the visualization appeared too late (\textit{Communication}). Based on this, we speculate that if the tests were conducted again without any rumble feedback for either hand, the \textit{Intuitiveness} score might have been similar for both versions. 

Focusing on the interviews now, we can see that the fingers got a lot of attention from the users. Many of them expressed liking the stroking and the finger adjustment while hovering over movable objects with the Adaptive Hands. Other users disliked that the Adaptive Hands could grab objects in ways that would make it impossible to get a grip of a similar object in reality (e.g. grabbing by touching only one flat surface). Similarly, others wished that the grip of the Adaptive Hands was tighter and always made contact with the grabbed objects. Some testers also expressly disliked that the Standard Hands became invisible while holding objects. These observations point to a general appreciation of the sense of touch conveyed by the Adaptive Hand and a desire to round it off by adjusting the fingers and perhaps even the hand position and rotation so that the grip is more believable.

\begin{figure}[h]
\centering
\includegraphics[height=6cm]{ImpossibleGrip.png}
\caption{Image showing a position where the hand seems unlikely to be able to grab the object, but has done so nonetheless.}
\label{fig:impossibleGrip}
\end{figure}

In the broad \textit{Physics} category, many testers appreciated that the Adaptive Hands made manipulating objects and hold them in their virtual hands without using the "artificial grabbing mechanic" controlled by the trigger buttons. Less users than we anticipated explicitly valued that the Adaptive Hands did not penetrate other virtual immovable objects. This might be again caused by the users having spent less time interacting with immovable objects than with movable objects, which the Standard Hands also don't penetrate in most use cases. As discussed in Chapter \ref{chap:theory}, PI, Psi and SoE are all subjective experiences and the degree to which each VR user will experience them depends entirely on how they interact with the environment. If they don't experience the cases where the system doesn't work, the illusions will not be broken. In regards to the physics, which we related to Psi, this means that the illusion worked similarly for both hand versions because the users weren't probing the limitations of the Standard Hand. Job Simulator relies often on this knowledge to avoid having to find technical solutions for certain problems, directing the attention of users to the areas of the game where their system works best.

A few of the subjects described unconsciously behaving as if they had virtual arms or a virtual torso only to realize that they didn't in fact have them. These situations strongly point to users experiencing a strong sense of PI and we think these situations happen often when the users are holding objects with their virtual hands naturally, without using the grabbing mechanic. An example of the situations we are describing could be he following: the user tries to hold an object with their virtual hands and intuitively tries to hold it against their chest for support, only to see the object fall as they realize that they have no virtual chest. If this kind of experience is in fact connected to the \textit{Natural Grab}, it would seem that this is an area with a lot of potential for improving PI and perhaps SoE (if this happens when the user actually has the expected virtual body parts).

One of the main concerns when considering implementing an adjusted hand control model to deal with the Virtual Constraints Problem is whether the separation between the user's real hands and their virtual hands would negatively impact the experience of the users (presumably because of loss of the SoE). In this regard, the results from the tests are inconclusive: the same amount of testers reported that the separation bothered them as the ones that explicitly mentioned that the separation didn't bother them, either by explaining that they didn't notice it or even by indicating that they felt like they would rather have a more believable grip on the objects they grabbed even if it meant that the virtual hands would separate from their own real hands. The subjects that expressed a disliking when the virtual hands separated from their real hands were not bothered by the adjustment of the fingers, which they found "natural", in this regard it seems like there was consensus among the testers. The hand position and rotation of the virtual hands is only different from the position and rotation of the real hands when the real hands are inside of virtual objects, but this is also the situation where most users found the rumble to be unpleasant, so this could have influenced their judgment (some users described the rumble as "small shocks" and others identified the rumble in these cases as a signal from the game telling them that they weren't supposed to be doing what they were doing). Furthermore, it is also in these cases when the stability issues sometimes happened, so improvements in this aspect as well could change the experience for the users. 

In sum, the results are not as clear-cut as one might have hoped: we see two possible interpretations of them. The simplest interpretation is to take the results of the A/B questions as the testers' overall opinion of the hand versions, which would indicate that they clearly preferred the Adaptive Hands. The rest of the results (description questions and interviews) in this case show us that there are still many issues with the adaptive hands that could be improved. A second reading of the results, however, is to consider that despite the A/B questions providing favorable results for the Adaptive Hands, the descriptions and the interviews are what actually reveals the testers' experience with each hand version, and in case of contradiction, the A/B results should be disregarded. 

Very few testers made observations about the Standard Hands or found any fault with them in the interviews. This could be due to the way the tests were setup, which set the focus on the Adaptive Hands. The testers always tried the Standard Hands first, and then the Adaptive Hands, instead of trying them in a randomized order. The names we used to refer to each version of the hands also framed the Adaptive Hand as an upgrade with respect to the Standard Hands. The hands could have been called "hand A" and "hand B", but we decided against it, fearing that it would make it harder to communicate with the testers.

If one were to follow Occam's Razor, one should find the first explanation to be more likely to be true, but we would still consider the second one as a cautionary note.

\todo[inline]{Controlled experiments were too unfocused. Compared versions should have been only different one sense: whether they were unadjusted or adjusted hand control models?
Test environments could be iterated on in order to increase the frequency with which the users engaged in the actions that we wanted them to experience.}